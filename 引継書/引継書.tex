\documentclass[dvipdfmx,11pt]{jarticle}   	
\usepackage[top=24truemm,bottom=24truemm,left=20truemm,right=20truemm]{geometry}
\usepackage{amsmath}
\usepackage[]{multicol}
\usepackage{titlesec}
\usepackage{fancyhdr}
\usepackage[dvipdfmx]{graphicx}
\usepackage[dvipdfmx]{color}
\usepackage{color}
\usepackage {fancybox}
\usepackage{jvlisting, listings}
\usepackage[deluxe]{otf}
\usepackage{graphicx}
\titleformat*{\section}{\large\bfseries}

\title{第62回聖光祭 技術部門 引き継ぎ書} 
\author{飼沼 隼 \\ 三枝 義啓 \and 浅井 祐一朗 \\ 矢向 俊貴 \and 李 博之 \\ 黒羽 }
\date{2021年 10 月 9 日} 

\begin{document}

\begin{center}

\textbf {
\vspace{8.5cm}
\\
\Ovalbox{\fontsize{33pt}{20pt}\selectfont \scalebox{1.05}[1]{超機密}}\\
\fontsize{50pt}{60pt}\selectfont \scalebox{1.05}[1]{技術部門引継書}\\
\fontsize{12pt}{30pt}\selectfont \scalebox{1.05}[1]{聖光祭実行委員会 拡大実行会議}\\
\fontsize{35pt}{35pt}\selectfont \scalebox{1.05}[1]{第1次最終報告}\\
\fontsize{12pt}{25pt}\selectfont \scalebox{1.05}[1]{聖光祭技術部門}\\
\fontsize{12pt}{15pt}\selectfont \scalebox{1.05}[1]{2021年度業務内容概要}\\
\fontsize{12pt}{20pt}\selectfont \scalebox{1.05}[1]{総括篇}\\
}
\end{center}
\pagebreak
\tableofcontents
\clearpage
\section{はじめに}
\subsection{挨拶}
これをお読みの方、こんにちは。第62回聖光祭技術副部門長の李博之\footnote{60227li@seiko.ac.jp}といいます。僕が代表してこの挨拶を書かせていただいていますが、もちろん本書はいろんな人が書いています。さて、本書は2021年10月2日〜同4日に開催された第62回聖光祭において技術部門がどのようなことを行なってきたか、そしてその反省をまとめた引き継ぎ書です。技術部門は他の部門と違い、毎回同じことをする部門ではありません。個人のスキルや人材の量と質に大きく作用されます。個人的にはみなさんができる範囲でベストを尽くして欲しいです。僕はこの時の聖光祭でアプリ局長、HP局長、副部門長などを兼任しており、アプリ局iOS班のリードプログラマでした。また、QRコードシステムと統一シフトを個人的に担当していました。忙しいことこの上ないです。そうゆう観点では人事もかなり重要になります。多忙なひとをできるだけ作らないようにしましょう。その人が足を引っ張ることになります。自分語りはこのくらいにしてそろそろ本題に移ろうと思います。この引き継ぎ書がどのくらい続くのかはわかりませんが、後輩の皆さんには絶大な期待と信頼を寄せています。この部門を仲間とともに立ち上げた身としてはできるだけみなさんの後悔がないようにすべてを終え、次に引き継いでもらいたいです。頑張って下さい。
\subsection{本書の取り扱い}
この引き継ぎ書は技術部門及び技術局発足年に書かれたものであり、基本的なことが多く乗っています。ですので、この引き継ぎ書は引き継ぐことをおすすめします。その年々によって修正や改変、追加などがあればTexファイルを直接編集するか、それを訂正する形で引き継ぎ書を書くのがいいと思います。また、その年あったことを体験談として書くのもいいと思うので別途準備した方が良さそうです。ですが、この引き継ぎ書で全てがわかる状態にするのであればリンクを貼り付けたりするのがいいと思います。
\subsection{Githubと情報公開}
今年から技術部門及び技術局でGithubOrganizationを運営し、全てのコードをGithubで管理することになりました。プログラマにとってコード整理やバージョン管理は重要な課題であり、聖光祭実行委員会と生徒会の透明化のため情報公開は必要だと考えています。Githubの使い方はAPPENDIXにて解説しています。管理者権限等が引き継がれなかった場合、先代の管理者または 60227li@seiko.ac.jp までご連絡ください。
\subsection{技術部門の位置付け}
「技術部門」は59期斎藤先輩が主導で60期の初期メンバーとともに立ち上げた部門です。それ以前は「総合技術研究所」(以下、総技研という)として外務の傘下にあり、アプリ局・HP局・動画局にわかれて活動していました。ですが、時代の変化や権限及び予算の関係で規模が大きくなり、生徒会や体育祭、そして学校での依頼や活動が増えた結果、パンフ局・デザイン課とともに「ITEC」として生徒会直属の部署となりました。聖光祭ではこのITECを特別に「技術部門」と呼称し、部門の一つとなりました。
\section{ITEC}
 \subsection{機材}
 現在動画局(生徒会)で所有している機材は以下の通りである。
 \begin{itemize}
  \item BlackmagicDesign Pocket Cinema Camera 4K\\
  広報委員会が生徒会予算で購入している。
  \item DJI RSC2\\
  カメラ用ジンバル。動画のプロジェクトで必要な機材として、工藤校長先生に購入していただいた。
  \item グリーンバック\\
  使おうと思って購入したが、結局使っていない。ぜひ使って欲しい。
  \item Windows Desktop PC (自作)\\
 40万くらいかけて自作したPC。CGのレンダリングに最適。
 \end{itemize}
\section{動画局}
\subsection{ソフト}
 動画を作る上で必要なアプリケーションを紹介しておく。
 \begin{itemize}
  \item Adobe系\\
  学校から支給されているものを使えば良いと思う。
  \item Davinci Resolve Studio\\
  カメラを購入した際についてきたものが、生徒会室のPCに割り当てられている。普通に買うと3万円くらいする。カラーグレーディングに最適。
  \item Cinema 4D\\
  CGソフト。40万くらいする映画にも使われているソフトだが、学生は実質無料\footnote{別の会社のプラットフォームを使って購入するため数百円の手数料が取られる}で使える。
  \item Octane\\
  レンダラー。Cinema 4Dと共に使うととてもいい映像が出来上がる。
  有料であるが、予算で落として使うべし。\\ただし、1,2アカウントを何人かで使うことになり、同時に使用することができないので注意が必要。\\生徒会室のPCをレンダリング専用に割り当てたいので、2アカウント購入すべし。
 \end{itemize}
\subsection{動画}
最低限作るべき動画は以下の通り。
 \begin{itemize}
  \item スローガン動画\\
  フルCG。動画
  \item 幹部紹介動画
  \item オープニング動画
  \item グランドフィナーレオープニング動画
  \item エンディング動画
 \end{itemize}
 
 \section{HP局}
 62回局長:李博之(60227)
 \subsection{ソフト}
 あくまで僕が使用したソフトである。htmlとcssとjsをいじれるならなんだっていい。
  \begin{itemize}
  \item Visual Studio Code\\
  エディタ。自分好みに機能やデザインをカスタムできるため、汎用性が高い。Microsoft社製でもちろん無料。同じようなものでGithubが出してるAtomというものもあり、個人的にはAtomの方がデザインはいいと思う。VSCodeにAtomizeというプラグインを入れてAtom風にしていた。プラグインは後ほどまとめる。
  \item Adobe XD\\
  UI/UX設計ソフト。他のAdobeソフトとの互換性がとてもいい配置ソフト。オブジェクトを配置してサイトのUIを検討することができる。直感的に操作でき、画面遷移もシュミレートできる。ただし、CSSやHTMLを書き出してくれるわけではないので必須ではない。個人的には仮でもUIのイメージがあると作りやすいので必ず使っている。
  \item Adobe Dreamweaver\\
  ウェブアプリ制作の統合型開発環境的なもの。初期段階で使うことを検討していた。ウェブ制作のすべてがここにある。リアルタイムのUIシュミレートやサーバーへの更新設定、サイトの公開もやってくれる神ソフトだが、重いのが難点。パソコンを変えたときにインストールを忘れていて自然消滅した。正直全く不満はなかった。
  \item その他Adobeソフト\\
  デザイン局が忙しそうだったので素材は全部自分で作った。楽しい。illustratorは使えないと恥ずかしい。
  \item FileZilla\\
  いわゆるFTPクライアントソフトというもの。FTPサーバーにファイルをアップロードするときに使う。FFFTPとか有名どころは他にもある。すべてUIが古いWindows感でてて怪しいが全然普通のソフト。FTPについては後ほど解説する。
  \end{itemize}
 \section{アプリ局}
 \section{デザイン局}
 \section{さいごに}
 \section{APPENDIX}
 \subsection{Githubの使い方}
\end{document}