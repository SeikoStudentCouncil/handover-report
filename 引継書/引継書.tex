\documentclass[dvipdfmx,ja4,11pt,tittlepage]{jsarticle}
\usepackage[dvipdfmx]{graphicx}
\usepackage{pdfpages}


\title{引き継ぎ書} 
\author{聖光学院生徒会 ITEC}
\date{2021年度} 
\begin{document}
\maketitle
\newpage
\section{ITEC}
 \subsection{機材}
 現在動画局(生徒会)で所有している機材は以下の通りである。
 \begin{itemize}
 \item BlackmagicDesign Pocket Cinema Camera 4K\\
 広報委員会が生徒会予算で購入している。
 \item DJI RSC2\\
 カメラ用ジンバル。動画のプロジェクトで必要な機材として、工藤校長先生に購入していただいた。
 \item グリーンバック\\
 使おうと思って購入したが、結局使っていない。ぜひ使って欲しい。
 \item Windows Desktop PC (自作)\\
 40万くらいかけて自作したPC。CGのレンダリングに最適。
 \end{itemize}
\section{動画局}
\subsection{ソフト}
 動画を作る上で必要なアプリケーションを紹介しておく。
 \begin{itemize}
 \item Adobe系
 学校から支給されているものを使えば良いと思う。
 \item Davinci Resolve Studio\\
 カメラを購入した際についてきたものが、生徒会室のPCに割り当てられている。普通に買うと3万円くらいする。カラーグレーディングに最適。
 \item Cinema 4D\\
 CGソフト。40万くらいする映画にも使われているソフトだが、学生は実質無料\footnote{別の会社のプラットフォームを使って購入するため数百円の手数料が取られる}で使える。
 \item Octane\\
 レンダラー。Cinema 4Dと共に使うととてもいい映像が出来上がる。
 有料であるが、予算で落として使うべし。\\ただし、1,2アカウントを何人かで使うことになり、同時に使用することができないので注意が必要。\\生徒会室のPCをレンダリング専用に割り当てたいので、2アカウント購入すべし。
 \end{itemize}
 \subsection{}
\end{document}